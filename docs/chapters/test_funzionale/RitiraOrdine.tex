\section{RitiraOrdine}

\subsubsection*{Category Partition Testing}

\begin{table}[!hbp]
	\centering
	\footnotesize
	\begin{tblr}{
		colspec = XXXXXX,
		hlines, vlines,
		row{1} = {font=\bfseries},
		measure=vbox, stretch=-1
		}
		idOrdine \\
		\begin{itemize}[leftmargin=*]
			\item Stringa di caratteri di lunghezza $\leq$ 45
			\item Stringa di caratteri di lunghezza $>$ 45 \texttt{[ERROR]}
			\item Stringa di caratteri vuota \texttt{[ERROR]}
		\end{itemize}
	\end{tblr}
\end{table}

\noindent Essendo previsto un solo input, il numero di test da effettuarsi è pari a 3.

\subsubsection*{Test Suite}

\begin{table}[!hbp]
	\centering
	\footnotesize
	\begin{tblr}{
			colspec = lXXXlXX,
			hlines, vlines,
			row{1} = {font=\bfseries},
			measure=vbox
		}
		{Test \\ Case \\ ID} & Descrizione & Classi di Equivalenza Coperte & Pre-condizioni & Input & {Output \\ Attesi} & {Post-condizioni \\ Attese} \\
		1 & ID dell'ordine valido & ID dell'ordine valido & - & ID : 5ea930bc-f0a5-427a-8ca1-f9a2a6146948 & Stato ordine cambiato con successo & - \\
		2 & ID dell'ordine $>$ 45 caratteri & ID dell'ordine $>$ 45 caratteri \texttt{[ERROR]} & - & ID : ... & ID dell'ordine troppo lungo & - \\
		3 & ID dell'ordine assente & ID dell'ordine assente \texttt{[ERROR]} & - & ID : & Inserire l'ID di un ordine & - \\
	\end{tblr}
\end{table}