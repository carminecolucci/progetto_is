\section{CercaFarmaco}

\subsubsection*{Category Partition Testing}

\begin{table}[!hbp]
	\centering
	\footnotesize
	\begin{tblr}{
		colspec = XXXXXX,
		hlines, vlines,
		row{1} = {font=\bfseries},
		measure=vbox, stretch=-1
		}
		Nome \\
		\begin{itemize}[leftmargin=*]
			\item Stringa di caratteri di lunghezza $\leq$ 45
			\item Stringa di caratteri di lunghezza $>$ 45 \texttt{[ERROR]}
			\item Stringa di caratteri vuota \texttt{[ERROR]}
		\end{itemize}
	\end{tblr}
\end{table}

\noindent Essendo previsto un solo input, il numero di test da effettuarsi è pari a 3.

\subsubsection*{Test Suite}

\begin{table}[!hbp]
	\centering
	\footnotesize
	\begin{tblr}{
			colspec = lXXXlXX,
			hlines, vlines,
			row{1} = {font=\bfseries},
			measure=vbox
		}
		{Test \\ Case \\ ID} & Descrizione & Classi di Equivalenza Coperte & Pre-condizioni & Input & {Output \\ Attesi} & {Post-condizioni \\ Attese} \\
		1 & Nome del farmaco valido & Nome del farmaco valido & - & Nome : Tachipirina & Ricerca effettuata & - \\
		2 & Nome del farmaco $>$ 45 caratteri & Nome del farmaco $>$ 45 caratteri \texttt{[ERROR]} & - & Nome : ... & Nome del farmaco troppo lungo & - \\
		3 & Nome del farmaco assente & Nome del farmaco assente \texttt{[ERROR]} & - & Nome : & Inserire il nome di un farmaco & - \\
	\end{tblr}
\end{table}