\section{RegistraCliente}

\subsubsection*{Category Partition Testing}

\begin{table}[!hbp]
	\centering
	\footnotesize
	\begin{tblr}{
		colspec = XXXXXX,
		hlines, vlines,
		row{1} = {font=\bfseries},
		measure=vbox, stretch=-1
		}
		Nome & Cognome & Username & Password & Email & DataNascita \\
		\begin{itemize}[leftmargin=*]
			\item Stringa di caratteri di lunghezza $\leq$ 45
			\item Stringa di caratteri di lunghezza $>$ 45 \texttt{[ERROR]}
			\item Stringa di caratteri vuota \texttt{[ERROR]}
		\end{itemize} &
		\begin{itemize}[leftmargin=*]
			\item Stringa di caratteri di lunghezza $\leq$ 45
			\item Stringa di caratteri di lunghezza $>$ 45 \texttt{[ERROR]}
			\item Stringa di caratteri vuota \texttt{[ERROR]}
		\end{itemize} &
		\begin{itemize}[leftmargin=*]
			\item Stringa di caratteri di lunghezza $\leq$ 45
			\item Stringa di caratteri di lunghezza $>$ 45 \texttt{[ERROR]}
			\item Stringa di caratteri vuota \texttt{[ERROR]}
		\end{itemize} &
		\begin{itemize}[leftmargin=*]
			\item Stringa di caratteri di lunghezza compresa tra 8 e 45
			\item Stringa di caratteri di lunghezza $<$ 8 \texttt{[ERROR]}
			\item Stringa di caratteri di lunghezza $>$ 45 \texttt{[ERROR]}
		\end{itemize} &
		\begin{itemize}[leftmargin=*]
			\item Stringa di caratteri di lunghezza $\leq$ 45
			\item Stringa di caratteri di lunghezza $>$ 45 \texttt{[ERROR]}
			\item Stringa di caratteri vuota \texttt{[ERROR]}
		\end{itemize} &
		\begin{itemize}[leftmargin=*]
			\item Data con formato valido (gg-mm-aaaa)
			\item Data con formato non valido \texttt{[ERROR]}
		\end{itemize}
	\end{tblr}
\end{table}

\noindent Il numero di test da effettuarsi senza particolari vincoli è: $3 \cdot 3 \cdot 3 \cdot 3 \cdot 3 \cdot 2 = 486$.

\noindent Introduciamo i vincoli \texttt{[ERROR]}. Il numero di test da eseguire per testare singolarmente i vincoli è 11 (2 per Nome, 2 per Cognome, 2 per Username, 2 per Password, 2 per Email, 1 per DataNascita).

\noindent Il numero di test risultante è 12: $(1 \cdot 1 \cdot 1 \cdot 1 \cdot 1 \cdot 1) + 11 = 12$.

\subsubsection*{Test Suite}

\begin{table}[!hbp]
	\centering
	\footnotesize
	\begin{tblr}{
			colspec = lXXXlXX,
			hlines, vlines,
			row{1} = {font=\bfseries},
			measure=vbox
		}
		{Test \\ Case \\ ID} & Descrizione & Classi di Equivalenza Coperte & Pre-condizioni & Input & {Output \\ Attesi} & {Post-condizioni \\ Attese} \\
		1 &
		Tutti gli input validi &
		Nome, Cognome, Username, Password, Email, DataNascita validi &
		Il cliente non è ancora registrato nel sistema &
		{Nome: Mario \\ Cognome: Rossi \\ Username: mariorossi \\ Password: miapassword \\ Email: mariorossi@gmail.com \\ DataNascita: 22-06-1989} &
		Registrazione effettuata & Il cliente è stato correttamente registrato nel sistema \\
		2 &
		Nome $>$ 45 caratteri &
		Nome $>$ 45 caratteri \texttt{[ERROR]}, Cognome, Username, Password, Email, DataNascita validi &
		- &
		{Nome: \dots \\ Cognome: Rossi \\ Username: mariorossi \\ Password: miapassword \\ Email: mariorossi@gmail.com \\ DataNascita: 22-06-1989} &
		Nome troppo lungo &
		- \\
		3 &
		Nome assente &
		Nome assente \texttt{[ERROR]}, Cognome, Username, Password, Email, DataNascita validi &
		- &
		{Nome: \\ Cognome: Rossi \\ Username: mariorossi \\ Password: miapassword \\ Email: mariorossi@gmail.com \\ DataNascita: 22-06-1989} &
		Inserire un nome &
		- \\
	\end{tblr}
\end{table}

\begin{table}[!hbp]
	\centering
	\footnotesize
	\begin{tblr}{
			colspec = lXXXlXX,
			hlines, vlines,
			row{1} = {font=\bfseries},
			measure=vbox
		}
		{Test \\ Case \\ ID} & Descrizione & Classi di Equivalenza Coperte & Pre-condizioni & Input & {Output \\ Attesi} & {Post-condizioni \\ Attese} \\
		4 &
		Cognome $>$ 45 caratteri &
		Cognome $>$ 45 caratteri \texttt{[ERROR]}, Nome, Username, Password, Email, DataNascita validi &
		- &
		{Nome: Mario \\ Cognome: \dots \\ Username: mariorossi \\ Password: miapassword \\ Email: mariorossi@gmail.com \\ DataNascita: 22-06-1989} &
		Cognome troppo lungo &
		- \\
		5 &
		Cognome assente &
		Cognome assente \texttt{[ERROR]}, Nome, Username, Password, Email, DataNascita validi &
		- &
		{Nome: Mario \\ Cognome: \\ Username: mariorossi \\ Password: miapassword \\ Email: mariorossi@gmail.com \\ DataNascita: 22-06-1989} &
		Inserire un cognome &
		- \\
		6 &
		Username $>$ 45 caratteri &
		Username $>$ 45 caratteri \texttt{[ERROR]}, Nome, Cognome, Password, Email, DataNascita validi &
		- &
		{Nome: Mario \\ Cognome: Rossi \\ Username: \dots \\ Password: miapassword \\ Email: mariorossi@gmail.com \\ DataNascita: 22-06-1989} &
		Username troppo lungo &
		- \\
		7 &
		Username assente &
		Username assente \texttt{[ERROR]}, Nome, Cognome, Password, Email, DataNascita validi &
		- &
		{Nome: Mario \\ Cognome: Rossi \\ Username: \\ Password: miapassword \\ Email: mariorossi@gmail.com \\ DataNascita: 22-06-1989} &
		Inserire un username &
		- \\
		8 &
		Password $<$ 8 caratteri &
		Password $<$ 8 caratteri \texttt{[ERROR]}, Nome, Cognome, Username, Email, DataNascita validi &
		- &
		{Nome: Mario \\ Cognome: Rossi \\ Username: mariorossi \\ Password: prova \\ Email: mariorossi@gmail.com \\ DataNascita: 22-06-1989} &
		Password troppo corta &
		- \\
		9 &
		Password $>$ 45 caratteri &
		Password $>$ 45 caratteri \texttt{[ERROR]}, Nome, Cognome, Username, Email, DataNascita validi &
		- &
		{Nome: Mario \\ Cognome: Rossi \\ Username: mariorossi \\ Password: \dots \\ Email: mariorossi@gmail.com \\ DataNascita: 22-06-1989} &
		Password troppo lunga &
		- \\
		10 &
		Email $>$ 45 caratteri &
		Email $>$ 45 caratteri \texttt{[ERROR]}, Nome, Cognome, Username, Password, DataNascita validi &
		- &
		{Nome: Mario \\ Cognome: Rossi \\ Username: mariorossi \\ Password: miapassword \\ Email: \dots \\ DataNascita: 22-06-1989} &
		Email troppo lunga &
		- \\
		11 &
		Email assente &
		Email assente \texttt{[ERROR]}, Nome, Cognome, Username, Password, DataNascita validi &
		- &
		{Nome: Mario \\ Cognome: Rossi \\ Username: \\ Password: miapassword \\ Email: \\ DataNascita: 22-06-1989} &
		Inserire un'email &
		- \\
		12 &
		Formato DataNascita non valida &
		Formato DataNascita non valida \texttt{[ERROR]}, Nome, Cognome, Username, Password, Email validi &
		- &
		{Nome: Mario \\ Cognome: Rossi \\ Username: mariorossi \\ Password: miapassword \\ Email: mariorossi@gmail.com \\ DataNascita: 1989-06-22} &
		Formato data non valido &
		- \\
	\end{tblr}
\end{table}
