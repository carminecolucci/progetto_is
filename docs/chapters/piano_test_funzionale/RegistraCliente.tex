\section{RegistraCliente}

\subsubsection*{Category Partition Testing}

\begin{table}[H]
	\centering
	\footnotesize
	\begin{partest}{colspec = XXXXXX}
		Nome & Cognome & Username & Password & Email & DataNascita \\
		\begin{itemize}[leftmargin=*]
			\item Stringa di lunghezza $\leq$ 45
			\item Stringa di lunghezza $>$ 45 \texttt{[ERROR]}
			\item Stringa vuota \texttt{[ERROR]}
		\end{itemize} &
		\begin{itemize}[leftmargin=*]
			\item Stringa di lunghezza $\leq$ 45
			\item Stringa di lunghezza $>$ 45 \texttt{[ERROR]}
			\item Stringa vuota \texttt{[ERROR]}
		\end{itemize} &
		\begin{itemize}[leftmargin=*]
			\item Stringa di lunghezza $\leq$ 45
			\item Stringa di lunghezza $>$ 45 \texttt{[ERROR]}
			\item Stringa vuota \texttt{[ERROR]}
			\item Stringa già presente nel sistema \texttt{[ERROR]}
		\end{itemize} &
		\begin{itemize}[leftmargin=*]
			\item Stringa di lunghezza compresa tra 8 e 45
			\item Stringa di lunghezza $<$ 8 \texttt{[ERROR]}
			\item Stringa di lunghezza $>$ 45 \texttt{[ERROR]}
		\end{itemize} &
		\begin{itemize}[leftmargin=*]
			\item Stringa di lunghezza $\leq$ 45
			\item Stringa di lunghezza $>$ 45 \texttt{[ERROR]}
			\item Stringa vuota \texttt{[ERROR]}
			\item Stringa già presente nel sistema \texttt{[ERROR]}
		\end{itemize} &
		\begin{itemize}[leftmargin=*]
			\item Data con formato valido (gg-mm-aaaa)
			\item Data con formato non valido \texttt{[ERROR]}
		\end{itemize}
	\end{partest}
\end{table}

\noindent Il numero di test da effettuarsi senza particolari vincoli è: $3 \cdot 3 \cdot 4 \cdot 3 \cdot 4 \cdot 2 = 864$.

\noindent Introduciamo i vincoli \texttt{[ERROR]}. Il numero di test da eseguire per testare singolarmente i vincoli è 13 (2 per Nome, 2 per Cognome, 3 per Username, 2 per Password, 3 per Email, 1 per DataNascita).

\noindent Il numero di test risultante è 14: $(1 \cdot 1 \cdot 1 \cdot 1 \cdot 1 \cdot 1) + 13 = 14$.

\subsubsection*{Test Suite}

\begin{table}[H]
	\centering
	\footnotesize
	\begin{testsuite}{colspec = lXXXlXX}
		{Test \\ Case \\ ID} & Descrizione & Classi di Equivalenza Coperte & Pre-condizioni & Input & {Output \\ Attesi} & {Post-condizioni \\ Attese} \\
		1 & Tutti gli input validi & Nome, Cognome, Username, Password, Email, DataNascita validi & Il cliente non è ancora registrato nel sistema & {Nome: Mario \\ Cognome: Rossi \\ Username: mariorossi \\ Password: miapassword \\ Email: mario@gmail.com \\ DataNascita: 22-06-1989} & Registrazione effettuata & Il cliente è stato correttamente registrato nel sistema \\
		2 & Nome $>$ 45 caratteri & Nome $>$ 45 caratteri \texttt{[ERROR]}, Cognome, Username, Password, Email, DataNascita validi & -- & {Nome: \dots \\ Cognome: Rossi \\ Username: mariorossi \\ Password: miapassword \\ Email: mario@gmail.com \\ DataNascita: 22-06-1989} & Nome troppo lungo & -- \\
		3 & Nome non specificato & Nome non specificato \texttt{[ERROR]}, Cognome, Username, Password, Email, DataNascita validi & -- & {Nome: \\ Cognome: Rossi \\ Username: mariorossi \\ Password: miapassword \\ Email: mario@gmail.com \\ DataNascita: 22-06-1989} & Inserire un nome & -- \\
	\end{testsuite}
\end{table}

\begin{table}[H]
	\centering
	\footnotesize
	\begin{testsuite}{colspec = lXXXlXl}
		{Test \\ Case \\ ID} & Descrizione & Classi di Equivalenza Coperte & Pre-condizioni & Input & {Output \\ Attesi} & {Post-\\condizioni \\ Attese} \\
		4 & Cognome $>$ 45 caratteri & Cognome $>$ 45 caratteri \texttt{[ERROR]}, Nome, Username, Password, Email, DataNascita validi & -- & {Nome: Mario \\ Cognome: \dots \\ Username: mariorossi \\ Password: miapassword \\ Email: mario@gmail.com \\ DataNascita: 22-06-1989} & Cognome troppo lungo & -- \\
		5 & Cognome non specificato & Cognome non specificato \texttt{[ERROR]}, Nome, Username, Password, Email, DataNascita validi & -- & {Nome: Mario \\ Cognome: \\ Username: mariorossi \\ Password: miapassword \\ Email: mario@gmail.com \\ DataNascita: 22-06-1989} & Inserire un cognome & -- \\
		6 & Username $>$ 45 caratteri & Username $>$ 45 caratteri \texttt{[ERROR]}, Nome, Cognome, Password, Email, DataNascita validi & -- & {Nome: Mario \\ Cognome: Rossi \\ Username: \dots \\ Password: miapassword \\ Email: mario@gmail.com \\ DataNascita: 22-06-1989} & Username troppo lungo & -- \\
		7 & Username non specificato & Username non specificato \texttt{[ERROR]}, Nome, Cognome, Password, Email, DataNascita validi & -- & {Nome: Mario \\ Cognome: Rossi \\ Username: \\ Password: miapassword \\ Email: mario@gmail.com \\ DataNascita: 22-06-1989} & Inserire uno username & -- \\
		8 & Password $<$ 8 caratteri & Password $<$ 8 caratteri \texttt{[ERROR]}, Nome, Cognome, Username, Email, DataNascita validi & -- & {Nome: Mario \\ Cognome: Rossi \\ Username: mariorossi \\ Password: prova \\ Email: mario@gmail.com \\ DataNascita: 22-06-1989} & Password troppo corta & -- \\
		9 & Password $>$ 45 caratteri & Password $>$ 45 caratteri \texttt{[ERROR]}, Nome, Cognome, Username, Email, DataNascita validi & -- & {Nome: Mario \\ Cognome: Rossi \\ Username: mariorossi \\ Password: \dots \\ Email: mario@gmail.com \\ DataNascita: 22-06-1989} & Password troppo lunga & -- \\
		10 & Email $>$ 45 caratteri & Email $>$ 45 caratteri \texttt{[ERROR]}, Nome, Cognome, Username, Password, DataNascita validi & -- & {Nome: Mario \\ Cognome: Rossi \\ Username: mariorossi \\ Password: miapassword \\ Email: \dots \\ DataNascita: 22-06-1989} & Email troppo lunga & -- \\
		11 & Email non specificata & Email non specificato \texttt{[ERROR]}, Nome, Cognome, Username, Password, DataNascita validi & -- & {Nome: Mario \\ Cognome: Rossi \\ Username: \\ Password: miapassword \\ Email: \\ DataNascita: 22-06-1989} & Inserire un'email & -- \\
		12 & Formato DataNascita non valido & Formato DataNascita non valido \texttt{[ERROR]}, Nome, Cognome, Username, Password, Email validi & -- & {Nome: Mario \\ Cognome: Rossi \\ Username: mariorossi \\ Password: miapassword \\ Email: mario@gmail.com \\ DataNascita: 1989-06-22} & Formato data non valido & -- \\
	\end{testsuite}
\end{table}

\begin{table}[H]
	\centering
	\footnotesize
	\begin{testsuite}{colspec = lXXXlll}
		{Test \\ Case \\ ID} & Descrizione & Classi di Equivalenza Coperte & Pre-condizioni & Input & {Output \\ Attesi} & {Post-\\condizioni \\ Attese} \\
		13 & Username già presente nel sistema & Username già esistente \texttt{[ERROR]}, Nome, Cognome, Password, DataNascita ed Email validi & Username ``pippo2002'' già presente nel sistema & {Nome: Pippo \\ Cognome: Baudo \\ Username: pippo2002 \\ Password: miapassword \\ Email: pippo@gmail.com \\ DataNascita: 1989-06-22} & {Username già \\ utilizzato} & -- \\
		14 & Email già presente nel sistema & Email già esistente \texttt{[ERROR]}, Nome, Cognome, Password, DataNascita e Username validi & Email ``pippo@gmail.com'' già presente nel sistema & {Nome: Pippo \\ Cognome: Baudo \\ Username: pippo2002 \\ Password: miapassword \\ Email: pippo@gmail.com \\ DataNascita: 1989-06-22} & {Email già \\ utilizzata} & -- \\
	\end{testsuite}
\end{table}
