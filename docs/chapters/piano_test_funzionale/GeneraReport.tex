\section{GeneraReport}

\subsubsection*{Category Partition Testing}

\begin{table}[H]
	\centering
	\footnotesize
	\begin{partest}{colspec = XX}
		DataInizio & DataFine\\
		\begin{itemize}[leftmargin=*]
			\item Data con formato valido (gg-mm-aaaa)
			\item Data con formato non valido \texttt{[ERROR]}
		\end{itemize} &
		\begin{itemize}[leftmargin=*]
			\item Data con formato valido (gg-mm-aaaa)
			\item Data con formato non valido \texttt{[ERROR]}
			\item DataFine precede DataInizio \texttt{[ERROR]}
		\end{itemize} \\
	\end{partest}
\end{table}

\noindent Il numero di test da effettuarsi senza particolari vincoli è: $2 \cdot 3 = 6$.

\noindent Introduciamo i vincoli \texttt{[ERROR]}. Il numero di test da eseguire per testare singolarmente i vincoli è 3 (1 per DataInizio, 2 per DataFine).

\noindent Il numero di test risultante è 4: $(1 \cdot 1) + 3 = 4$.

\subsubsection*{Test Suite}

\begin{table}[H]
	\centering
	\footnotesize
	\begin{testsuite}{colspec = llXXX}
		{Test \\ Case \\ ID} & Descrizione & {Classi di Equivalenza \\ Coperte} & Input & Output Attesi \\
		1 & {Tutti gli input \\ validi} & DataInizio, DataFine validi & {DataInizio: 01-06-2024 \\ DataFine: 31-06-2024} & Il report viene generato \\
		2 & {Formato \\ DataInizio non valido} & Formato DataInizio non valido \texttt{[ERROR]} & {DataInizio: 2024-06-01 \\ DataFine: 31-06-2024} & Formato DataInizio non valido \\
		3 & {Formato DataFine \\ non valido} & Formato DataFine non valido \texttt{[ERROR]} & {DataInizio: 01-06-2024 \\ DataFine: 2024-06-31} & Formato DataFine non valido \\
		4 & {DataFine precede \\ DataInizio} & DataFine precede DataInizio \texttt{[ERROR]} & {DataInizio: 31-06-2024 \\ DataFine: 01-06-2024} & Intervallo di date errato \\
	\end{testsuite}
\end{table}
