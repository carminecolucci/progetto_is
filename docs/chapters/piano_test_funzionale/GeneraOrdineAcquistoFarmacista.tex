\section{GeneraOrdineAcquistoFarmacista}

\begin{table}[!hbp]
	\centering
	\footnotesize
	\begin{partest}{colspec = XXXl}
		Farmaci-Quantità \\
		\begin{itemize}[leftmargin=*]
			\item Lista di coppie (nomeFarmaco, quantità) non vuota. Per ogni nomeFarmaco esiste un corrispondente farmaco nel sistema.
			\item Lista vuota \texttt{[ERROR]}
		\end{itemize}
	\end{partest}
\end{table}

\noindent Essendo previsto un solo input, il numero di test da effettuarsi è pari a 2.

\subsubsection*{Test Suite}

\begin{table}[!hbp]
	\centering
	\footnotesize
	\begin{testsuite}{colspec = lXXXlXX}
		{Test \\ Case \\ ID} & Descrizione & Classi di Equivalenza Coperte & Pre-condizioni & Input & {Output \\ Attesi} & {Post-condizioni \\ Attese} \\
		1 & Ordine valido & Farmaci-Quantità valido & Esistono nel sistema i farmaci 'Tachipirina' e 'Fluifort' & {[('Tachipirina', 5),\\ ('Fluifort', 10)]} & Ordine di acquisto generato & Un ordine di acquisto viene correttamente creato \\
		2 & Ordine vuoto & Lista vuota \texttt{[ERROR]} & -- & -- & Non puoi creare un ordine di acquisto vuoto & -- \\
	\end{testsuite}
\end{table}
