\section{LoginUtente}

\subsubsection*{Category Partition Testing}

\begin{table}[!hbp]
	\centering
	\footnotesize
	\begin{tblr}{
		colspec = XX,
		hlines, vlines,
		row{1} = {font=\bfseries},
		measure=vbox, stretch=-1
		}
		Username & Password \\
		\begin{itemize}[leftmargin=*]
			\item Stringa di lunghezza $\leq$ 45
			\item Stringa di lunghezza $>$ 45 \texttt{[ERROR]}
			\item Stringa vuota \texttt{[ERROR]}
		\end{itemize} &
		\begin{itemize}[leftmargin=*]
			\item Stringa di lunghezza compresa tra 8 e 45 e corrispondente alla Password memorizzate per l'utente che cerca di effettuare il login
			\item Stringa di lunghezza $<$ 8 \texttt{[ERROR]}
			\item Stringa di lunghezza $>$ 45 \texttt{[ERROR]}
			\item Stringa non corrispondente alla Password memorizzata per l'utente che cerca di effettuare il login \texttt{[ERROR]}
		\end{itemize}
	\end{tblr}
\end{table}

\noindent Il numero di test da effettuarsi senza particolari vincoli è: $3 \cdot 4 = 12$.

\noindent Introduciamo i vincoli \texttt{[ERROR]}. Il numero di test da eseguire per testare singolarmente i vincoli è 5 (2 per Username, 3 per Password).

\noindent Il numero di test risultante è 6: $(1 \cdot 1) + 5 = 6$.

\subsubsection*{Test Suite}

\begin{table}[!hbp]
	\centering
	\footnotesize
	\begin{tblr}{
			colspec = lXXXXXX,
			hlines, vlines,
			row{1} = {font=\bfseries},
			measure=vbox
		}
		{Test \\ Case \\ ID} & Descrizione & Classi di Equivalenza Coperte & Pre-condizioni & Input & {Output \\ Attesi} & {Post-condizioni \\ Attese} \\
		1 &
		Tutti gli input validi &
		Username, Password validi &
		{L'utente deve essere \\ correttamente \\ registrato nel sistema} &
		{Username: mariorossi \\ Password: miapassword} &
		Login effettuato & L'utente è entrato correttamente nel sistema \\
		2 &
		Username $>$ 45 caratteri &
		Username $>$ 45 caratteri \texttt{[ERROR]}, Password valida &
		--- &
		{Username: \dots \\ Password: miapassword} &
		Username troppo lungo &
		-- \\
		3 &
		Username assente &
		Username assente \texttt{[ERROR]}, Password valida &
		--- &
		{Username: \\ Password: miapassword} &
		Inserire un username &
		-- \\
		4 &
		Password $<$ 8 caratteri &
		Password $<$ 8 caratteri \texttt{[ERROR]}, Username valido &
		--- &
		{Username: mariorossi \\ Password: prova} &
		Password troppo corta &
		-- \\
		5 &
		Password $>$ 45 caratteri &
		Password $>$ 45 caratteri \texttt{[ERROR]}, Username valido &
		--- &
		{Username: mariorossi \\ Password: \dots} &
		Password troppo lunga &
		-- \\
		6 &
		Password valida ma errata &
		Password errata \texttt{[ERROR]}, Username valido &
		L'utente esiste nel sistema e ha come password `passwd' &
		{Username: mariorossi \\ Password: ciao} &
		Password errata &
		-- \\
	\end{tblr}
\end{table}
